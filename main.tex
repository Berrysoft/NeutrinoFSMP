\documentclass[aspectratio=149]{beamer}
\def\mathfamilydefault{\rmdefault}
\usepackage[heading]{ctex}
\usepackage{subfigure,color,bm}
\usepackage{mathtools}
\usepackage{multicol}
\usepackage{mhchem}
\usepackage{tikz}
\usetikzlibrary{positioning, shapes.geometric}
\usetheme{berlin}

\title{GPU 上的中微子探测}
\author{Berrysoft(王宇逸)}
\begin{document}
\begin{frame}
    \titlepage
\end{frame}
\section{物理背景}
\begin{frame}
    \frametitle{何为中微子}

    \begin{multicols}{2}
        在一个典型的$\beta$衰变中,原子核放出一个电子。
        这个电子的能量服从右图的分布。

        由于能量动量守恒,泡利假设存在另一种粒子分走了部分能量动量。
        费米将其命名为\textbf{中微子}(neutrino)。

        \begin{equation*}
            n\to p+e^-+\bar{\nu}_e
        \end{equation*}
        \columnbreak
        \begin{figure}
            \centering
            \includegraphics[width=0.5\textwidth]{RaE1.jpg}
            \caption{图片来源:维基百科}
        \end{figure}
    \end{multicols}

\end{frame}

\begin{frame}
    \frametitle{中微子质量顺序}

    中微子振荡的发现确认了中微子存在质量,这是一个超出现有标准模型的结论。
    确定三个质量本征态的质量顺序,能够帮助我们排除一些物理模型。

    如今我们能够确定$m_2>m_1,\Delta m_{31}^2 \gg \Delta m_{21}^2$。
    因此中微子质量仅存在两种排序可能:
    $m_3>m_2>m_1$(正序)或$m_2>m_1>m_3$(反序)。

\end{frame}

\begin{frame}[allowframebreaks]
    \frametitle{江门中微子实验(JUNO)}

    \begin{figure}
        \centering
        \includegraphics[height=0.6\textheight]{juno_det.png}
        \caption{JUNO 的探测器结构。图片来源:百度百科}
    \end{figure}

    \begin{figure}
        \centering
        \includegraphics[height=0.6\textheight]{juno_geo.png}
        \caption{JUNO 的地理位置。图片来源:百度百科}
    \end{figure}

\end{frame}

\begin{frame}
    \frametitle{JUNO 的物理目标}

    \begin{multicols}{2}
        我们实际上能够探测的是电子反中微子$\bar{\nu}_e$,
        它的能谱根据中微子质量顺序的不同而有着微弱的差异,如右图。

        如果能够测量确认实际的能谱属于哪一种,就能够确认中微子的质量顺序。
        \columnbreak
        \begin{figure}
            \centering
            \includegraphics[width=0.5\textwidth]{nu_energy.png}
        \end{figure}
    \end{multicols}

\end{frame}
\section{探测器物理过程}
\begin{frame}
    \frametitle{逆$\beta$衰变}

    液体闪烁体中含有大量的质子,中微子主要和质子反应:
    \begin{equation*}
        \bar{\nu}_e+p\to n+e^+
    \end{equation*}
    产生的正电子会与液闪分子相互作用,产生大量光子。
    正电子会与电子湮灭:
    \begin{equation*}
        e^+ +e^-\to\gamma+\gamma
    \end{equation*}
    放出的$\gamma$主要与电子发生康普顿散射,被散射的电子也会与液闪分子相互作用产生光子。

\end{frame}

\begin{frame}
    \frametitle{光电倍增管}

    \begin{multicols}{2}
        光电倍增管(PMT)是一个信号放大器。
        它利用光电效应,将光子信号转变为单电子信号。
        再将一个电子变成许多电子,从而形成可以观测的波形。

        \begin{figure}
            \centering
            \includegraphics[width=0.3\textwidth]{pmt_work.png}
            \caption{PMT 的放大原理示意图}
        \end{figure}
        \columnbreak
        \begin{figure}
            \centering
            \includegraphics[height=0.7\textheight]{pmt.jpg}
        \end{figure}
    \end{multicols}

\end{frame}

\begin{frame}
    \frametitle{输出波形}

    PMT 输出的波形除了电子信号,还有电子学噪声。

    \begin{figure}
        \centering
        \includegraphics[width=0.5\textwidth]{draw2_4.png}
        \caption{一个 PMT 上可能的波形示意,垂直线是真实的电子信号时间}
    \end{figure}

\end{frame}
\section{波形分析}
\begin{frame}
    \frametitle{经典重建算法}



\end{frame}

\begin{frame}
    \frametitle{马尔科夫链蒙特卡罗}



\end{frame}

\begin{frame}
    \frametitle{混合MCMC链}



\end{frame}
\section{GPU 加速}
\begin{frame}
    \frametitle{手写 CUDA(失败)}



\end{frame}

\begin{frame}
    \frametitle{批量处理}



\end{frame}

\begin{frame}
    \frametitle{CuPy 与 Tensorflow 的速度对比}

    \begin{figure}
        \centering
        \includegraphics[width=0.75\textwidth]{bench.pdf}
    \end{figure}

\end{frame}
\section{展望未来}
\begin{frame}
    \frametitle{能量重建、多点源重建与粒子鉴别}



\end{frame}

\begin{frame}
    \frametitle{致谢}



\end{frame}
\section{参考文献}

\end{document}
